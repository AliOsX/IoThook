%% Generated by Sphinx.
\def\sphinxdocclass{report}
\documentclass[letterpaper,10pt,turkish]{sphinxmanual}
\ifdefined\pdfpxdimen
   \let\sphinxpxdimen\pdfpxdimen\else\newdimen\sphinxpxdimen
\fi \sphinxpxdimen=.75bp\relax

\usepackage[utf8]{inputenc}
\ifdefined\DeclareUnicodeCharacter
 \ifdefined\DeclareUnicodeCharacterAsOptional\else
  \DeclareUnicodeCharacter{00A0}{\nobreakspace}
\fi\fi
\usepackage{cmap}
\usepackage[T1]{fontenc}
\usepackage{amsmath,amssymb,amstext}
\usepackage{babel}
\usepackage{times}
\usepackage[Sonny]{fncychap}
\usepackage{longtable}
\usepackage{sphinx}

\usepackage{geometry}
\usepackage{multirow}
\usepackage{eqparbox}

% Include hyperref last.
\usepackage{hyperref}
% Fix anchor placement for figures with captions.
\usepackage{hypcap}% it must be loaded after hyperref.
% Set up styles of URL: it should be placed after hyperref.
\urlstyle{same}

\addto\captionsturkish{\renewcommand{\figurename}{Şekil}}
\addto\captionsturkish{\renewcommand{\tablename}{Tablo}}
\addto\captionsturkish{\renewcommand{\literalblockname}{Liste}}

\addto\extrasturkish{\def\pageautorefname{sayfa}}





\title{IoTHook Documentation}
\date{May 31, 2017}
\release{1.2}
\author{electrocoder}
\newcommand{\sphinxlogo}{}
\renewcommand{\releasename}{Sürüm}
\makeindex

\begin{document}
\ifnum\catcode`\=\string=\active\shorthandoff{=}\fi
\maketitle
\sphinxtableofcontents
\phantomsection\label{\detokenize{index::doc}}



\chapter{Quick reference}
\label{\detokenize{index:iot-hook}}\label{\detokenize{index:icindekiler}}

\section{Iot Nedir}
\label{\detokenize{what-is-iot:iot-nedir}}\label{\detokenize{what-is-iot::doc}}\label{\detokenize{what-is-iot:what-is-iot}}

\subsection{Iot Nedir?}
\label{\detokenize{what-is-iot:id1}}
Nesnelerin interneti \sphinxquotedblleft{}internet of things\sphinxquotedblright{} 1999 yılında Kevin Ashton tarafından kullanılan bir kavramdır
ve teknolojideki gelişmeler ile birlikte bugünkü haline gelmiştir. RFID teknolojisi için
üretilen bu kavram günümüzde tüm elektronik cihazlara uygulanmaktadır.
\begin{figure}[htbp]
\centering
\capstart

\noindent\sphinxincludegraphics[scale=0.65]{{internet-of-things}.png}
\caption{\sphinxstyleemphasis{IoT - Nesnelerin İnterneti}}\label{\detokenize{what-is-iot:id2}}\end{figure}


\subsection{Iothook nedir?}
\label{\detokenize{what-is-iot:iothook-nedir}}
Iothook internete bağlı nesneler (iot) arasında veri transferi yapan web servis
ağı projesidir. Iothook ile Arduino, Raspberry Pi, Android, iOS, Windows Phone, Web Site, Banana Pi,
Orange Pi, Beaglebone, ARM, Pic, Windows, Mac OS X, ve Linux tabanlı sistemleri birbirine bağlar.


\subsection{Niçin Iothook?}
\label{\detokenize{what-is-iot:nicin-iothook}}\begin{itemize}
\item {} 
Iothook hızlıdır,

\item {} 
Sınırsız kanal oluşturabilirsin,

\item {} 
Sınırsız element ekleyebilirsin,

\item {} 
Tüm cihazların ile kolayca veri gönderebilirsin (post),

\item {} 
Tüm iot cihazlarından kolayca veri alabilirsin (get),

\item {} 
Datalarını gerçek zamanlı takip edebilirsin,

\item {} 
Dataların için gerçek zamanlı grafik oluşturabilirsin,

\end{itemize}


\subsection{Iothook’ un sunduğu avantajlar:}
\label{\detokenize{what-is-iot:iothook-un-sundugu-avantajlar}}\begin{itemize}
\item {} 
Kanal oluşturma,

\item {} 
Kanal elementi ekleme,

\item {} 
Web api,

\item {} 
Web Sorgu,

\item {} 
Form api,

\item {} 
Twit atma,

\item {} 
SMS atma,

\item {} 
E-posta,

\item {} 
Grafik,

\item {} 
7/24 destek,

\end{itemize}

Iothook tüm cihazlarınız arasında kesintisiz veri aktarımı yapan, internete bağlı nesnelerin
kolayca ulaşabileceği iletişim protokollerini destekler.

Google developer chart apileri ile entegre olarak verileri gerçek zamanlı izleme olanağı sağlar.


\section{Kanal Aç}
\label{\detokenize{create-new-channel::doc}}\label{\detokenize{create-new-channel:create-new-channel}}\label{\detokenize{create-new-channel:kanal-ac}}
Iothook kanal; internete bağlı nesneler arasında veri iletimini sağlamak için oluşturulmuş
kanca sistemidir. Kanal ile iot sistemleri veri paylaşımı yapılabilir, veri gönderim işlemleri tanımlanır.

Iothook web servislerini kullanabilmek için üye olunmalıdır. Üyelik
seçenekleri ‘Free’, ‘Student’, ‘Pro’ ve ‘Ultra’ olmak üzere 4
kullanım planı vardır. Üye olmak için adrese gidiniz.

Üyelik adımından sonra yönetim paneli aracılığı ile ‘Kanal Ekle’ ekranına girilir.


\subsection{Kanal Ekle}
\label{\detokenize{create-new-channel:kanal-ekle}}
Kanal ekleme adımları şu şekildedir:
\begin{itemize}
\item {} \begin{description}
\item[{Form Metod: Http (Hyper Text Transfer Protocol) de veriler TCP/IP metodu ile iletilmektedir. Http protokolü üzerinden veri iletimi request ve response istekleri ile gerçekleşir. ‘Request’ gerçekleşmesi istenen     talep, ‘Response’ ise yanıt olarak kullanılır. HTTP protokolüne göre POST, GET veya POST/GET metodu seçilir. Iothook iletişiminde post ve get metodları kullanılmaktadır.}] \leavevmode\begin{itemize}
\item {} 
Post: Verilerin iot cihazda mesaj gövdesine yerleştirilerek gönderilme işlemidir.

\item {} 
Get: Verilerin iot cihaz ile sorgulanma ve cevap alınma talebidir.

\item {} 
Post/Get: Veri aktarımının iot nesnesi ile server arasında çift taraflı olacağını gösterir.

\end{itemize}

\end{description}

\item {} 
Form enctype: \sphinxquotedblleft{}application/x-www-form-urlencoded\sphinxquotedblright{} ile iot cihazından gönderilen karakterlerin gönderilmeden önce kodlanacağını belirtir. \sphinxquotedblleft{}mutlipart/form-data\sphinxquotedblright{} ise verilerin içerisinde ASCII olmayan verilerin olduğunu dosya veya image formatında veri olduğunu belirtir.

\item {} 
Aygıt türü: Iot cihazın türünü belirler. Arduino, Raspberry Pi... gibi

\item {} 
Kanal adı: Verilerin toplanacağı kanalın adı.

\item {} 
Web site: Veriler bir web sitesinde kullanılacak ise web site adresi girilmelidir.

\item {} 
Email ile haber ver: Veri alındığında kayıtlı olan mail adresine mesaj gonderir. Aktif edilirse 15dk. da bir veri gönderilmesi gerekir.

\item {} 
Verileri kaydet: Iot nesnesinden gelen verilerin iothook veritabanında saklanması için gereklidir.

\item {} 
Resim: Kanal tanıtım resmi olarak kullanılır.

\item {} 
Açıklama: Kanal bilgileri girilmelidir.

\item {} 
Is public POST: Bu kanal genel kullanıma açık ve veri eklenmesine açıktır.

\item {} 
Is public GET: Bu kanal genel kullanıma açık ve verilerin okunmasına izin verir.

\item {} 
Yayındamı: Kanalı aktif et.

\end{itemize}


\subsection{Element Ekle}
\label{\detokenize{create-new-channel:element-ekle}}
Iot cihazınız için kanal oluşturduktan sonra kanalda bulunmasını istediğiniz
veri alanlarını oluşturmalısınız. Bu alanlar veri almaya başlamak
için eklenir. Element verilerine POST veta GET metodu ile ulaşabilirsiniz.


\subsection{Element ayarları:}
\label{\detokenize{create-new-channel:element-ayarlari}}\begin{itemize}
\item {} 
Kanal adı: Elementin hangi kanala veri aktaracağı seçilir.

\item {} 
Grafik türü: Toplanan verilerin çizileceği grafik türünü belirler.

\item {} 
Element tipi: Verilerin depolanacağı alan tipini belirler. Grafik çizimi sadece \sphinxquotedblleft{}number\sphinxquotedblright{} veri tipinde yapılmalıdır.

\item {} 
Kanal adı: Verilerin toplanacağı kanalın adı.

\item {} 
Element adı: Verilerin tutulacağı element adı.

\item {} 
Yayındamı: Elementi aktif et.

\end{itemize}


\section{Veri Gönder}
\label{\detokenize{send-data::doc}}\label{\detokenize{send-data:veri-gonder}}\label{\detokenize{send-data:send-data}}
Veri göndermek için öncelikle kanal ve element eklemeniz gerekir. Kanal oluşturulduğunda
size özel \sphinxquotedblleft{}api\_key\sphinxquotedblright{} üretilerek belirlenen erişim metoduna göre (POST, GET, POST/GET) veri işlemi gerçekleştirilir.

Örneğin; Kanalımız ısı, ışık, hareket, bar ve nem değerlerini alan bir yapıda olsun.
Kanal içerisinde bulunacak iot cihazlarımız bizlere bu dataları 15 sn. yede bir 100 kere göndersin.

Oluşturulan \sphinxquotedblleft{}API\_KEY\sphinxquotedblright{} Key Yöneticisi sayfasından görülebilir.


\subsection{Python 2 Json ile Veri Gönderme}
\label{\detokenize{send-data:python-2-json-ile-veri-gonderme}}
Python Json ile Post Örneği:

Bu örneği \sphinxurl{https://goo.gl/v9Gd3U} sayfasından indirebilirsiniz.

\begin{sphinxVerbatim}[commandchars=\\\{\}]
\PYG{l+s+sd}{\PYGZdq{}\PYGZdq{}\PYGZdq{}}
\PYG{l+s+sd}{  Python 2 ile IoThook REST Api Testi}

\PYG{l+s+sd}{  Kod çalıştırıldığında auth kullanıcı adı ve şifre ile doğrulama gerçekleştirilir.}
\PYG{l+s+sd}{  Kanal api\PYGZus{}key ile ilgili kanal ve element değerleri IoThook a post edilir.}

\PYG{l+s+sd}{  Bu ornek IotHook servisine veri almak/gondermek icin baslangic seviyesinde}
\PYG{l+s+sd}{  testlerin yapilmasini amaclamaktadir.}

\PYG{l+s+sd}{  10 Mayıs 2017}
\PYG{l+s+sd}{  Sahin MERSIN}

\PYG{l+s+sd}{  Daha fazlasi icin}

\PYG{l+s+sd}{  http://www.iothook.com}
\PYG{l+s+sd}{  ve}
\PYG{l+s+sd}{  https://github.com/electrocoder/iotHook}

\PYG{l+s+sd}{  sitelerine gidiniz.}

\PYG{l+s+sd}{  Sorular ve destek talepleri icin}
\PYG{l+s+sd}{  https://github.com/electrocoder/iotHook/issues}
\PYG{l+s+sd}{  sayfasindan veya Meşe Bilişim den yardım alabilirsiniz.}

\PYG{l+s+sd}{  Yayin : http://mesebilisim.com}

\PYG{l+s+sd}{  Licensed under the Apache License, Version 2.0 (the \PYGZdq{}License\PYGZdq{}).}
\PYG{l+s+sd}{  You may not use this file except in compliance with the License.}
\PYG{l+s+sd}{  A copy of the License is located at}

\PYG{l+s+sd}{  http://www.apache.org/licenses/}

\PYG{l+s+sd}{\PYGZdq{}\PYGZdq{}\PYGZdq{}}

\PYG{k+kn}{import} \PYG{n+nn}{requests}
\PYG{k+kn}{import} \PYG{n+nn}{json}
\PYG{k+kn}{import} \PYG{n+nn}{urllib}
\PYG{k+kn}{import} \PYG{n+nn}{urllib2}
\PYG{k+kn}{import} \PYG{n+nn}{random}
\PYG{k+kn}{import} \PYG{n+nn}{pprint}
\PYG{k+kn}{import} \PYG{n+nn}{time}


\PYG{n}{headers} \PYG{o}{=} \PYG{p}{\PYGZob{}}\PYG{l+s+s1}{\PYGZsq{}}\PYG{l+s+s1}{Content\PYGZhy{}type}\PYG{l+s+s1}{\PYGZsq{}}\PYG{p}{:} \PYG{l+s+s1}{\PYGZsq{}}\PYG{l+s+s1}{application/json}\PYG{l+s+s1}{\PYGZsq{}}\PYG{p}{\PYGZcb{}}
\PYG{n}{url} \PYG{o}{=} \PYG{l+s+s1}{\PYGZsq{}}\PYG{l+s+s1}{https://iothook.com/api/v1.2/datas/}\PYG{l+s+s1}{\PYGZsq{}}

\PYG{n}{auth}\PYG{o}{=}\PYG{p}{(}\PYG{l+s+s1}{\PYGZsq{}}\PYG{l+s+s1}{anonymoususer}\PYG{l+s+s1}{\PYGZsq{}}\PYG{p}{,} \PYG{l+s+s1}{\PYGZsq{}}\PYG{l+s+s1}{a12345678}\PYG{l+s+s1}{\PYGZsq{}}\PYG{p}{)}

\PYG{k}{for} \PYG{n}{i} \PYG{o+ow}{in} \PYG{n+nb}{range}\PYG{p}{(}\PYG{l+m+mi}{100}\PYG{p}{)}\PYG{p}{:}
    \PYG{n}{data}\PYG{o}{=}\PYG{p}{\PYGZob{}}
        \PYG{l+s+s1}{\PYGZsq{}}\PYG{l+s+s1}{api\PYGZus{}key}\PYG{l+s+s1}{\PYGZsq{}}\PYG{p}{:}\PYG{l+s+s1}{\PYGZsq{}}\PYG{l+s+s1}{F6H8h7dnGggc\PYGZhy{}9c\PYGZhy{}R\PYGZhy{}UahcVV20wbsVg}\PYG{l+s+s1}{\PYGZsq{}}\PYG{p}{,}
        \PYG{l+s+s1}{\PYGZsq{}}\PYG{l+s+s1}{element\PYGZus{}1}\PYG{l+s+s1}{\PYGZsq{}}\PYG{p}{:}\PYG{l+s+s1}{\PYGZsq{}}\PYG{l+s+s1}{isi}\PYG{l+s+s1}{\PYGZsq{}}\PYG{p}{,} \PYG{l+s+s1}{\PYGZsq{}}\PYG{l+s+s1}{value\PYGZus{}1}\PYG{l+s+s1}{\PYGZsq{}}\PYG{p}{:}\PYG{n}{i}\PYG{o}{*}\PYG{l+m+mi}{10}\PYG{p}{,}
        \PYG{l+s+s1}{\PYGZsq{}}\PYG{l+s+s1}{element\PYGZus{}2}\PYG{l+s+s1}{\PYGZsq{}}\PYG{p}{:}\PYG{l+s+s1}{\PYGZsq{}}\PYG{l+s+s1}{isik}\PYG{l+s+s1}{\PYGZsq{}}\PYG{p}{,} \PYG{l+s+s1}{\PYGZsq{}}\PYG{l+s+s1}{value\PYGZus{}2}\PYG{l+s+s1}{\PYGZsq{}}\PYG{p}{:}\PYG{n}{i}\PYG{o}{*}\PYG{l+m+mi}{20}\PYG{p}{,}
        \PYG{l+s+s1}{\PYGZsq{}}\PYG{l+s+s1}{element\PYGZus{}3}\PYG{l+s+s1}{\PYGZsq{}}\PYG{p}{:}\PYG{l+s+s1}{\PYGZsq{}}\PYG{l+s+s1}{hareket}\PYG{l+s+s1}{\PYGZsq{}}\PYG{p}{,} \PYG{l+s+s1}{\PYGZsq{}}\PYG{l+s+s1}{value\PYGZus{}3}\PYG{l+s+s1}{\PYGZsq{}}\PYG{p}{:}\PYG{n}{i}\PYG{o}{*}\PYG{l+m+mi}{30}\PYG{p}{,}
        \PYG{l+s+s1}{\PYGZsq{}}\PYG{l+s+s1}{element\PYGZus{}4}\PYG{l+s+s1}{\PYGZsq{}}\PYG{p}{:}\PYG{l+s+s1}{\PYGZsq{}}\PYG{l+s+s1}{bar}\PYG{l+s+s1}{\PYGZsq{}}\PYG{p}{,} \PYG{l+s+s1}{\PYGZsq{}}\PYG{l+s+s1}{value\PYGZus{}4}\PYG{l+s+s1}{\PYGZsq{}}\PYG{p}{:}\PYG{n}{i}\PYG{o}{*}\PYG{l+m+mi}{40}\PYG{p}{,}
        \PYG{l+s+s1}{\PYGZsq{}}\PYG{l+s+s1}{element\PYGZus{}5}\PYG{l+s+s1}{\PYGZsq{}}\PYG{p}{:}\PYG{l+s+s1}{\PYGZsq{}}\PYG{l+s+s1}{nem}\PYG{l+s+s1}{\PYGZsq{}}\PYG{p}{,} \PYG{l+s+s1}{\PYGZsq{}}\PYG{l+s+s1}{value\PYGZus{}5}\PYG{l+s+s1}{\PYGZsq{}}\PYG{p}{:}\PYG{n}{i}\PYG{o}{*}\PYG{l+m+mi}{50}\PYG{p}{,}
        \PYG{p}{\PYGZcb{}}

    \PYG{n}{data\PYGZus{}json} \PYG{o}{=} \PYG{n}{json}\PYG{o}{.}\PYG{n}{dumps}\PYG{p}{(}\PYG{n}{data}\PYG{p}{)}
    \PYG{n}{response} \PYG{o}{=} \PYG{n}{requests}\PYG{o}{.}\PYG{n}{post}\PYG{p}{(}\PYG{n}{url}\PYG{p}{,} \PYG{n}{data}\PYG{o}{=}\PYG{n}{data\PYGZus{}json}\PYG{p}{,} \PYG{n}{headers}\PYG{o}{=}\PYG{n}{headers}\PYG{p}{,} \PYG{n}{auth}\PYG{o}{=}\PYG{n}{auth}\PYG{p}{)}
    \PYG{k}{print}\PYG{p}{(}\PYG{n}{response}\PYG{p}{)}
    \PYG{k}{print}\PYG{p}{(}\PYG{n}{response}\PYG{o}{.}\PYG{n}{json}\PYG{p}{(}\PYG{p}{)}\PYG{p}{)}
    \PYG{n}{time}\PYG{o}{.}\PYG{n}{sleep}\PYG{p}{(}\PYG{l+m+mi}{15}\PYG{p}{)}
\end{sphinxVerbatim}


\subsection{Python 3 Json ile Veri Gönderme}
\label{\detokenize{send-data:python-3-json-ile-veri-gonderme}}
Python Json ile Post Örneği:

Bu örneği \sphinxurl{https://goo.gl/7lyYV1} sayfasından inceliyebilirsiniz.

\begin{sphinxVerbatim}[commandchars=\\\{\}]
\PYG{c+c1}{\PYGZsh{} \PYGZhy{}*\PYGZhy{} coding: utf\PYGZhy{}8 \PYGZhy{}*\PYGZhy{}}

\PYG{l+s+sd}{\PYGZdq{}\PYGZdq{}\PYGZdq{}}
\PYG{l+s+sd}{  Python 3 ile IoThook REST Api Testi}

\PYG{l+s+sd}{  Kod çalıştırıldığında auth kullanıcı adı ve şifre ile doğrulama gerçekleştirilir.}
\PYG{l+s+sd}{  Kanal api\PYGZus{}key ile ilgili kanal ve element değerleri IoThook a post edilir.}

\PYG{l+s+sd}{  Bu ornek IotHook servisine veri almak/gondermek icin baslangic seviyesinde}
\PYG{l+s+sd}{  testlerin yapilmasini amaclamaktadir.}

\PYG{l+s+sd}{  10 Mayıs 2017}
\PYG{l+s+sd}{  Sahin MERSIN}

\PYG{l+s+sd}{  Daha fazlasi icin}

\PYG{l+s+sd}{  http://www.iothook.com}
\PYG{l+s+sd}{  ve}
\PYG{l+s+sd}{  https://github.com/electrocoder/iotHook}

\PYG{l+s+sd}{  sitelerine gidiniz.}

\PYG{l+s+sd}{  Sorular ve destek talepleri icin}
\PYG{l+s+sd}{  https://github.com/electrocoder/iotHook/issues}
\PYG{l+s+sd}{  sayfasindan veya Meşe Bilişim den yardım alabilirsiniz.}

\PYG{l+s+sd}{  Yayin : http://mesebilisim.com}

\PYG{l+s+sd}{  Licensed under the Apache License, Version 2.0 (the \PYGZdq{}License\PYGZdq{}).}
\PYG{l+s+sd}{  You may not use this file except in compliance with the License.}
\PYG{l+s+sd}{  A copy of the License is located at}

\PYG{l+s+sd}{  http://www.apache.org/licenses/}

\PYG{l+s+sd}{\PYGZdq{}\PYGZdq{}\PYGZdq{}}

\PYG{k+kn}{import} \PYG{n+nn}{requests}
\PYG{k+kn}{import} \PYG{n+nn}{json}
\PYG{k+kn}{import} \PYG{n+nn}{urllib}
\PYG{k+kn}{import} \PYG{n+nn}{random}
\PYG{k+kn}{import} \PYG{n+nn}{pprint}
\PYG{k+kn}{import} \PYG{n+nn}{time}

\PYG{n}{headers} \PYG{o}{=} \PYG{p}{\PYGZob{}}\PYG{l+s+s1}{\PYGZsq{}}\PYG{l+s+s1}{Content\PYGZhy{}type}\PYG{l+s+s1}{\PYGZsq{}}\PYG{p}{:} \PYG{l+s+s1}{\PYGZsq{}}\PYG{l+s+s1}{application/json}\PYG{l+s+s1}{\PYGZsq{}}\PYG{p}{\PYGZcb{}}
\PYG{n}{url} \PYG{o}{=} \PYG{l+s+s1}{\PYGZsq{}}\PYG{l+s+s1}{https://iothook.com/api/v1.2/datas/}\PYG{l+s+s1}{\PYGZsq{}}
\PYG{n}{auth}\PYG{o}{=}\PYG{p}{(}\PYG{l+s+s1}{\PYGZsq{}}\PYG{l+s+s1}{test}\PYG{l+s+s1}{\PYGZsq{}}\PYG{p}{,} \PYG{l+s+s1}{\PYGZsq{}}\PYG{l+s+s1}{test12345}\PYG{l+s+s1}{\PYGZsq{}}\PYG{p}{)}

\PYG{k}{for} \PYG{n}{i} \PYG{o+ow}{in} \PYG{n+nb}{range}\PYG{p}{(}\PYG{l+m+mi}{10}\PYG{p}{)}\PYG{p}{:}
    \PYG{n}{data}\PYG{o}{=}\PYG{p}{\PYGZob{}}
        \PYG{l+s+s1}{\PYGZsq{}}\PYG{l+s+s1}{api\PYGZus{}key}\PYG{l+s+s1}{\PYGZsq{}}\PYG{p}{:}\PYG{l+s+s1}{\PYGZsq{}}\PYG{l+s+s1}{f8c4a4d07a6\PYGZhy{}dc92f27f7b2}\PYG{l+s+s1}{\PYGZsq{}}\PYG{p}{,}
        \PYG{l+s+s1}{\PYGZsq{}}\PYG{l+s+s1}{element\PYGZus{}1}\PYG{l+s+s1}{\PYGZsq{}}\PYG{p}{:}\PYG{l+s+s1}{\PYGZsq{}}\PYG{l+s+s1}{sicaklik}\PYG{l+s+s1}{\PYGZsq{}}\PYG{p}{,} \PYG{l+s+s1}{\PYGZsq{}}\PYG{l+s+s1}{value\PYGZus{}1}\PYG{l+s+s1}{\PYGZsq{}}\PYG{p}{:}\PYG{n}{i}\PYG{o}{*}\PYG{l+m+mi}{100}\PYG{p}{,}
        \PYG{l+s+s1}{\PYGZsq{}}\PYG{l+s+s1}{element\PYGZus{}2}\PYG{l+s+s1}{\PYGZsq{}}\PYG{p}{:}\PYG{l+s+s1}{\PYGZsq{}}\PYG{l+s+s1}{isik}\PYG{l+s+s1}{\PYGZsq{}}\PYG{p}{,} \PYG{l+s+s1}{\PYGZsq{}}\PYG{l+s+s1}{value\PYGZus{}2}\PYG{l+s+s1}{\PYGZsq{}}\PYG{p}{:}\PYG{n}{i}\PYG{o}{*}\PYG{l+m+mi}{200}\PYG{p}{,}
        \PYG{l+s+s1}{\PYGZsq{}}\PYG{l+s+s1}{element\PYGZus{}3}\PYG{l+s+s1}{\PYGZsq{}}\PYG{p}{:}\PYG{l+s+s1}{\PYGZsq{}}\PYG{l+s+s1}{hareket}\PYG{l+s+s1}{\PYGZsq{}}\PYG{p}{,} \PYG{l+s+s1}{\PYGZsq{}}\PYG{l+s+s1}{value\PYGZus{}3}\PYG{l+s+s1}{\PYGZsq{}}\PYG{p}{:}\PYG{n}{i}\PYG{o}{*}\PYG{l+m+mi}{300}\PYG{p}{,}
        \PYG{l+s+s1}{\PYGZsq{}}\PYG{l+s+s1}{element\PYGZus{}4}\PYG{l+s+s1}{\PYGZsq{}}\PYG{p}{:}\PYG{l+s+s1}{\PYGZsq{}}\PYG{l+s+s1}{bar}\PYG{l+s+s1}{\PYGZsq{}}\PYG{p}{,} \PYG{l+s+s1}{\PYGZsq{}}\PYG{l+s+s1}{value\PYGZus{}4}\PYG{l+s+s1}{\PYGZsq{}}\PYG{p}{:}\PYG{n}{i}\PYG{o}{*}\PYG{l+m+mi}{400}\PYG{p}{,}
        \PYG{l+s+s1}{\PYGZsq{}}\PYG{l+s+s1}{element\PYGZus{}5}\PYG{l+s+s1}{\PYGZsq{}}\PYG{p}{:}\PYG{l+s+s1}{\PYGZsq{}}\PYG{l+s+s1}{nem}\PYG{l+s+s1}{\PYGZsq{}}\PYG{p}{,} \PYG{l+s+s1}{\PYGZsq{}}\PYG{l+s+s1}{value\PYGZus{}5}\PYG{l+s+s1}{\PYGZsq{}}\PYG{p}{:}\PYG{n}{i}\PYG{o}{*}\PYG{l+m+mi}{500}\PYG{p}{,}
        \PYG{p}{\PYGZcb{}}

    \PYG{n}{data\PYGZus{}json} \PYG{o}{=} \PYG{n}{json}\PYG{o}{.}\PYG{n}{dumps}\PYG{p}{(}\PYG{n}{data}\PYG{p}{)}

    \PYG{n}{response} \PYG{o}{=} \PYG{n}{requests}\PYG{o}{.}\PYG{n}{post}\PYG{p}{(}\PYG{n}{url}\PYG{p}{,} \PYG{n}{data}\PYG{o}{=}\PYG{n}{data\PYGZus{}json}\PYG{p}{,} \PYG{n}{headers}\PYG{o}{=}\PYG{n}{headers}\PYG{p}{,} \PYG{n}{auth}\PYG{o}{=}\PYG{n}{auth}\PYG{p}{)}
    \PYG{n}{pprint}\PYG{o}{.}\PYG{n}{pprint}\PYG{p}{(}\PYG{n}{response}\PYG{o}{.}\PYG{n}{json}\PYG{p}{(}\PYG{p}{)}\PYG{p}{)}
    \PYG{n}{time}\PYG{o}{.}\PYG{n}{sleep}\PYG{p}{(}\PYG{l+m+mi}{15}\PYG{p}{)}
\end{sphinxVerbatim}


\section{Veri Al}
\label{\detokenize{read-data::doc}}\label{\detokenize{read-data:veri-al}}\label{\detokenize{read-data:read-data}}
Iot cihazından gönderilen ısı, nem, voltaj, ışık gibi değerleri iothook data
merkezinden çekebilmek için öncelikle kanal üye kullanıcı adı ve giriş şifresineihtiyaç vardır. Iot cihazından gelen veriler, Android, iOS gibi mobil cihazınızdan
veya web sitenizden izlenebilir. Kanalınızın kullanımı genel kullanıma açık ise diğer kullanıcılar
ile de bu verileri paylaşabilirsiniz.

Örneğin; Kanal adımız \sphinxquotedblleft{}Temperature sensor\sphinxquotedblright{} olarak belirlenmiş ve kanal içerisinde bulunacak
iot cihazımızdan \sphinxquotedblleft{}temperature\sphinxquotedblright{} ve \sphinxquotedblleft{}humidity\sphinxquotedblright{} element verileri gönderiliyor olsun.


\subsection{Python 2, Python 3 Json ile Veri Alma}
\label{\detokenize{read-data:python-2-python-3-json-ile-veri-alma}}
Python Json ile Get Örneği:

Bu örneği \sphinxurl{https://goo.gl/Cd74oF} sayfasından inceleyebilirsiniz.

\begin{sphinxVerbatim}[commandchars=\\\{\}]
\PYG{c+c1}{\PYGZsh{} \PYGZhy{}*\PYGZhy{} coding: utf\PYGZhy{}8 \PYGZhy{}*\PYGZhy{}}

\PYG{l+s+sd}{\PYGZdq{}\PYGZdq{}\PYGZdq{}}
\PYG{l+s+sd}{  Python 2, Python 3 ile IoThook REST Api Testi}

\PYG{l+s+sd}{  Kod çalıştırıldığında \PYGZsq{}data\PYGZsq{} değişkenine verilen \PYGZsq{}all\PYGZsq{} değişkeni ile}
\PYG{l+s+sd}{  auth sahipliğindeki tüm veriler alınır.}

\PYG{l+s+sd}{  Bu ornek IotHook servisine veri almak/gondermek icin baslangic seviyesinde}
\PYG{l+s+sd}{  testlerin yapilmasini amaclamaktadir.}

\PYG{l+s+sd}{  10 Mayıs 2017}
\PYG{l+s+sd}{  Sahin MERSIN}

\PYG{l+s+sd}{  Daha fazlasi icin}

\PYG{l+s+sd}{  http://www.iothook.com}
\PYG{l+s+sd}{  ve}
\PYG{l+s+sd}{  https://github.com/electrocoder/iotHook}

\PYG{l+s+sd}{  sitelerine gidiniz.}

\PYG{l+s+sd}{  Sorular ve destek talepleri icin}
\PYG{l+s+sd}{  https://github.com/electrocoder/iotHook/issues}
\PYG{l+s+sd}{  sayfasindan veya Meşe Bilişim den yardım alabilirsiniz.}

\PYG{l+s+sd}{  Yayin : http://mesebilisim.com}

\PYG{l+s+sd}{  Licensed under the Apache License, Version 2.0 (the \PYGZdq{}License\PYGZdq{}).}
\PYG{l+s+sd}{  You may not use this file except in compliance with the License.}
\PYG{l+s+sd}{  A copy of the License is located at}

\PYG{l+s+sd}{  http://www.apache.org/licenses/}

\PYG{l+s+sd}{\PYGZdq{}\PYGZdq{}\PYGZdq{}}

\PYG{k+kn}{import} \PYG{n+nn}{requests}

\PYG{n}{url} \PYG{o}{=} \PYG{l+s+s1}{\PYGZsq{}}\PYG{l+s+s1}{https://iothook.com/api/v1.2/datas/?data=all}\PYG{l+s+s1}{\PYGZsq{}}

\PYG{n}{auth}\PYG{o}{=}\PYG{p}{(}\PYG{l+s+s1}{\PYGZsq{}}\PYG{l+s+s1}{test}\PYG{l+s+s1}{\PYGZsq{}}\PYG{p}{,} \PYG{l+s+s1}{\PYGZsq{}}\PYG{l+s+s1}{test12345}\PYG{l+s+s1}{\PYGZsq{}}\PYG{p}{)}

\PYG{n}{response} \PYG{o}{=} \PYG{n}{requests}\PYG{o}{.}\PYG{n}{get}\PYG{p}{(}\PYG{n}{url}\PYG{p}{,} \PYG{n}{auth}\PYG{o}{=}\PYG{n}{auth}\PYG{p}{)}
\PYG{n}{data} \PYG{o}{=} \PYG{n}{response}\PYG{o}{.}\PYG{n}{json}\PYG{p}{(}\PYG{p}{)}
\PYG{k}{print} \PYG{n}{data}
\end{sphinxVerbatim}


\subsection{Python 2, Python 3 Json ile İlk Veriyi Alma}
\label{\detokenize{read-data:python-2-python-3-json-ile-ilk-veriyi-alma}}
Python İlk Veriyi Alma, Json ile Get Örneği:

Bu örneği \sphinxurl{https://goo.gl/yNK75j} sayfasından inceleyebilirsiniz.

\begin{sphinxVerbatim}[commandchars=\\\{\}]
\PYG{l+s+sd}{\PYGZdq{}\PYGZdq{}\PYGZdq{}}
\PYG{l+s+sd}{  Python 2 ile IoThook REST Api Testi}

\PYG{l+s+sd}{  Kod çalıştırıldığında \PYGZsq{}data\PYGZsq{} değişkenine verilen \PYGZsq{}first\PYGZsq{} değişkeni ile}
\PYG{l+s+sd}{  auth sahipliğindeki ilk veri alınır. \PYGZsq{}channel\PYGZsq{} değişkeni Iothook dashboard}
\PYG{l+s+sd}{  Kanal oluşturma sırasında otomatik verilen id numarasıdır.}

\PYG{l+s+sd}{  Bu ornek IotHook servisine veri almak/gondermek icin baslangic seviyesinde}
\PYG{l+s+sd}{  testlerin yapilmasini amaclamaktadir.}

\PYG{l+s+sd}{  10 Mayıs 2017}
\PYG{l+s+sd}{  Sahin MERSIN}

\PYG{l+s+sd}{  Daha fazlasi icin}

\PYG{l+s+sd}{  http://www.iothook.com}
\PYG{l+s+sd}{  ve}
\PYG{l+s+sd}{  https://github.com/electrocoder/iotHook}

\PYG{l+s+sd}{  sitelerine gidiniz.}

\PYG{l+s+sd}{  Sorular ve destek talepleri icin}
\PYG{l+s+sd}{  https://github.com/electrocoder/iotHook/issues}
\PYG{l+s+sd}{  sayfasindan veya Meşe Bilişim den yardım alabilirsiniz.}

\PYG{l+s+sd}{  Yayin : http://mesebilisim.com}

\PYG{l+s+sd}{  Licensed under the Apache License, Version 2.0 (the \PYGZdq{}License\PYGZdq{}).}
\PYG{l+s+sd}{  You may not use this file except in compliance with the License.}
\PYG{l+s+sd}{  A copy of the License is located at}

\PYG{l+s+sd}{  http://www.apache.org/licenses/}

\PYG{l+s+sd}{\PYGZdq{}\PYGZdq{}\PYGZdq{}}

\PYG{k+kn}{import} \PYG{n+nn}{requests}

\PYG{n}{url} \PYG{o}{=} \PYG{l+s+s1}{\PYGZsq{}}\PYG{l+s+s1}{https://iothook.com/api/v1.2/datas/?data=first\PYGZam{}channel=106}\PYG{l+s+s1}{\PYGZsq{}}

\PYG{n}{auth}\PYG{o}{=}\PYG{p}{(}\PYG{l+s+s1}{\PYGZsq{}}\PYG{l+s+s1}{test}\PYG{l+s+s1}{\PYGZsq{}}\PYG{p}{,} \PYG{l+s+s1}{\PYGZsq{}}\PYG{l+s+s1}{test12345}\PYG{l+s+s1}{\PYGZsq{}}\PYG{p}{)}

\PYG{n}{response} \PYG{o}{=} \PYG{n}{requests}\PYG{o}{.}\PYG{n}{get}\PYG{p}{(}\PYG{n}{url}\PYG{p}{,} \PYG{n}{auth}\PYG{o}{=}\PYG{n}{auth}\PYG{p}{)}
\PYG{n}{data} \PYG{o}{=} \PYG{n}{response}\PYG{o}{.}\PYG{n}{json}\PYG{p}{(}\PYG{p}{)}
\PYG{k}{print} \PYG{n}{data}
\end{sphinxVerbatim}


\subsection{Python 2, Python 3 Json ile Son Veriyi Alma}
\label{\detokenize{read-data:python-2-python-3-json-ile-son-veriyi-alma}}
Python Son Veriyi Alma, Json ile Get Örneği:

Bu örneği \sphinxurl{https://goo.gl/iyvU7G} sayfasından inceleyebilirsiniz.

\begin{sphinxVerbatim}[commandchars=\\\{\}]
\PYG{l+s+sd}{\PYGZdq{}\PYGZdq{}\PYGZdq{}}
\PYG{l+s+sd}{  Python 2 ile IoThook REST Api Testi}

\PYG{l+s+sd}{  Kod çalıştırıldığında \PYGZsq{}data\PYGZsq{} değişkenine verilen \PYGZsq{}last\PYGZsq{} değişkeni ile}
\PYG{l+s+sd}{  auth sahipliğindeki en son veri alınır. \PYGZsq{}channel\PYGZsq{} değişkeni Iothook dashboard}
\PYG{l+s+sd}{  Kanal oluşturma sırasında otomatik verilen id numarasıdır.}

\PYG{l+s+sd}{  Bu ornek IotHook servisine veri almak/gondermek icin baslangic seviyesinde}
\PYG{l+s+sd}{  testlerin yapilmasini amaclamaktadir.}

\PYG{l+s+sd}{  10 Mayıs 2017}
\PYG{l+s+sd}{  Sahin MERSIN}

\PYG{l+s+sd}{  Daha fazlasi icin}

\PYG{l+s+sd}{  http://www.iothook.com}
\PYG{l+s+sd}{  ve}
\PYG{l+s+sd}{  https://github.com/electrocoder/iotHook}

\PYG{l+s+sd}{  sitelerine gidiniz.}

\PYG{l+s+sd}{  Sorular ve destek talepleri icin}
\PYG{l+s+sd}{  https://github.com/electrocoder/iotHook/issues}
\PYG{l+s+sd}{  sayfasindan veya Meşe Bilişim den yardım alabilirsiniz.}

\PYG{l+s+sd}{  Yayin : http://mesebilisim.com}

\PYG{l+s+sd}{  Licensed under the Apache License, Version 2.0 (the \PYGZdq{}License\PYGZdq{}).}
\PYG{l+s+sd}{  You may not use this file except in compliance with the License.}
\PYG{l+s+sd}{  A copy of the License is located at}

\PYG{l+s+sd}{  http://www.apache.org/licenses/}

\PYG{l+s+sd}{\PYGZdq{}\PYGZdq{}\PYGZdq{}}

\PYG{k+kn}{import} \PYG{n+nn}{requests}

\PYG{n}{url} \PYG{o}{=} \PYG{l+s+s1}{\PYGZsq{}}\PYG{l+s+s1}{https://iothook.com/api/v1.2/datas/?data=last\PYGZam{}channel=108}\PYG{l+s+s1}{\PYGZsq{}}

\PYG{n}{auth}\PYG{o}{=}\PYG{p}{(}\PYG{l+s+s1}{\PYGZsq{}}\PYG{l+s+s1}{test}\PYG{l+s+s1}{\PYGZsq{}}\PYG{p}{,} \PYG{l+s+s1}{\PYGZsq{}}\PYG{l+s+s1}{test12345}\PYG{l+s+s1}{\PYGZsq{}}\PYG{p}{)}

\PYG{n}{response} \PYG{o}{=} \PYG{n}{requests}\PYG{o}{.}\PYG{n}{get}\PYG{p}{(}\PYG{n}{url}\PYG{p}{,} \PYG{n}{auth}\PYG{o}{=}\PYG{n}{auth}\PYG{p}{)}
\PYG{n}{data} \PYG{o}{=} \PYG{n}{response}\PYG{o}{.}\PYG{n}{json}\PYG{p}{(}\PYG{p}{)}
\PYG{k}{print} \PYG{n}{data}
\end{sphinxVerbatim}


\subsection{Python 2, Python 3 Json ile Veriye Ait Detay Alma}
\label{\detokenize{read-data:python-2-python-3-json-ile-veriye-ait-detay-alma}}
Python veriye ait detay alma örneği:

Bu örneği \sphinxurl{https://goo.gl/Muvpbs} sayfasından inceleyebilirsiniz.

\begin{sphinxVerbatim}[commandchars=\\\{\}]
\PYG{l+s+sd}{\PYGZdq{}\PYGZdq{}\PYGZdq{}}
\PYG{l+s+sd}{  Python 2, 3 ile IoThook REST Api Testi}

\PYG{l+s+sd}{  Kod çalıştırıldığında datas url yapısına parametre olarak verilen}
\PYG{l+s+sd}{  değer Kanal ve Element içerisinde tanımlı datanın ayrıntılarını getirir.}

\PYG{l+s+sd}{  Bu ornek IotHook servisine veri almak/gondermek icin baslangic seviyesinde}
\PYG{l+s+sd}{  testlerin yapilmasini amaclamaktadir.}

\PYG{l+s+sd}{  10 Mayıs 2017}
\PYG{l+s+sd}{  Sahin MERSIN}

\PYG{l+s+sd}{  Daha fazlasi icin}

\PYG{l+s+sd}{  http://www.iothook.com}
\PYG{l+s+sd}{  ve}
\PYG{l+s+sd}{  https://github.com/electrocoder/iotHook}

\PYG{l+s+sd}{  sitelerine gidiniz.}

\PYG{l+s+sd}{  Sorular ve destek talepleri icin}
\PYG{l+s+sd}{  https://github.com/electrocoder/iotHook/issues}
\PYG{l+s+sd}{  sayfasindan veya Meşe Bilişim den yardım alabilirsiniz.}

\PYG{l+s+sd}{  Yayin : http://mesebilisim.com}

\PYG{l+s+sd}{  Licensed under the Apache License, Version 2.0 (the \PYGZdq{}License\PYGZdq{}).}
\PYG{l+s+sd}{  You may not use this file except in compliance with the License.}
\PYG{l+s+sd}{  A copy of the License is located at}

\PYG{l+s+sd}{  http://www.apache.org/licenses/}

\PYG{l+s+sd}{\PYGZdq{}\PYGZdq{}\PYGZdq{}}

\PYG{k+kn}{import} \PYG{n+nn}{requests}

\PYG{n}{url} \PYG{o}{=} \PYG{l+s+s1}{\PYGZsq{}}\PYG{l+s+s1}{https://iothook.com/api/v1.2/datas/8864/}\PYG{l+s+s1}{\PYGZsq{}}

\PYG{n}{auth}\PYG{o}{=}\PYG{p}{(}\PYG{l+s+s1}{\PYGZsq{}}\PYG{l+s+s1}{test}\PYG{l+s+s1}{\PYGZsq{}}\PYG{p}{,} \PYG{l+s+s1}{\PYGZsq{}}\PYG{l+s+s1}{test12345}\PYG{l+s+s1}{\PYGZsq{}}\PYG{p}{)}

\PYG{n}{response} \PYG{o}{=} \PYG{n}{requests}\PYG{o}{.}\PYG{n}{get}\PYG{p}{(}\PYG{n}{url}\PYG{p}{,} \PYG{n}{auth}\PYG{o}{=}\PYG{n}{auth}\PYG{p}{)}
\PYG{n}{data} \PYG{o}{=} \PYG{n}{response}\PYG{o}{.}\PYG{n}{json}\PYG{p}{(}\PYG{p}{)}
\PYG{k}{print} \PYG{n}{data}
\end{sphinxVerbatim}


\section{Email Besleme}
\label{\detokenize{email-feed:email-feed}}\label{\detokenize{email-feed::doc}}\label{\detokenize{email-feed:email-besleme}}

\subsection{Iot cihazlardan email alma}
\label{\detokenize{email-feed:iot-cihazlardan-email-alma}}
Iot cihazlardan email almak için `Kanal Ekle' menüsünden kanal oluşturulurken `Email feed'
seçeneğinin aktif edilmesi gerekir. Kanal oluşturulduktan sonrada email alma seçeneği
değiştirilebilir. Güncelleme için `Kanal Liste' menüsünden `Düzenle' seçeneği altından yapılabilir.


\subsection{Email besleme planı}
\label{\detokenize{email-feed:email-besleme-plani}}
Iot cihazınızdan veri geldiğinde email ile besleme almak için `STUDENT', `PRO',
veya `ULTRA' planlardan birisini tercih etmelisiniz.

Plan değişikliği için Ödeme sayfasından size uygun planı seçerek email besleme alabilirsiniz.


\subsection{Email besleme süresi}
\label{\detokenize{email-feed:email-besleme-suresi}}\begin{itemize}
\item {} 
Free plan email besleme süresi: 8 email, \textasciitilde{}180 dakika aralık ile

\item {} 
Student plan email besleme süresi: 10 email, \textasciitilde{}144 dakika aralık ile

\item {} 
Pro plan email besleme süresi: 15 email, \textasciitilde{}96 dakika aralık ile

\item {} 
Ultra plan email besleme süresi: 100 email, \textasciitilde{}14 dakika aralık ile

\end{itemize}


\section{Iot Email Sms Alarm}
\label{\detokenize{email-sms-alert::doc}}\label{\detokenize{email-sms-alert:iot-email-sms-alarm}}\label{\detokenize{email-sms-alert:email-sms-alert}}

\subsection{Alarm nedir?}
\label{\detokenize{email-sms-alert:alarm-nedir}}
Iot Kanal altında oluşturulan Elementlere alarm değeri kurma işlemidir. Alarm değeri kurularak iot cihazıdan her veri
alındığında operatör ile işlem yapılarak sonuca göre alarm üretilir. Üretilen alarm abonelik tipine göre bir günde en fazla
atılabilecek email ve sms planına göre belirlenir.


\subsection{Operatörler}
\label{\detokenize{email-sms-alert:operatorler}}\begin{description}
\item[{İşlem operatörleri aşağıdaki gibidir:}] \leavevmode\begin{itemize}
\item {} 
\textless{}   : Küçüktür operatörü. a \textless{} b. gelen\_deger \textless{} alarm\_degeri. Iot cihazdan gönderilen değer ile alarm değerini karşılaştırır. İşlem sonucu doğru (True) ise alarm üretilir.

\item {} 
\textless{}=  : Küçük eşittir operatörü. a \textless{}= b. gelen\_deger \textless{}= alarm\_degeri. Iot cihazdan gönderilen değer ile alarm değerini karşılaştırır. İşlem sonucu doğru (True) ise alarm üretilir.

\item {} 
==  : Eşittir operatörü. a == b. gelen\_deger == alarm\_degeri. Iot cihazdan gönderilen değer ile alarm değerini karşılaştırır. İşlem sonucu doğru (True) ise alarm üretilir.

\item {} 
!=  : Eşit değil operatörü. a != b. gelen\_deger != alarm\_degeri. Iot cihazdan gönderilen değer ile alarm değerini karşılaştırır. İşlem sonucu doğru (True) ise alarm üretilir.

\item {} 
\textgreater{}=  : Büyük eşit operatörü. a \textgreater{}= b. gelen\_deger \textgreater{}= alarm\_degeri. Iot cihazdan gönderilen değer ile alarm değerini karşılaştırır. İşlem sonucu doğru (True) ise alarm üretilir.

\item {} 
\textgreater{}   : Büyüktür operatörü. a \textgreater{} b. gelen\_deger \textgreater{} alarm\_degeri. Iot cihazdan gönderilen değer ile alarm değerini karşılaştırır. İşlem sonucu doğru (True) ise alarm üretilir.

\end{itemize}

\end{description}

Örnek operatör işlemleri:
\begin{quote}
\begin{itemize}
\item {} 
\textless{} Küçüktür operatörü python örnek:

\end{itemize}

\begin{sphinxVerbatim}[commandchars=\\\{\}]
\PYG{n}{a} \PYG{o}{=} \PYG{l+m+mi}{5}
\PYG{n}{b} \PYG{o}{=} \PYG{l+m+mi}{7}
\PYG{n}{a} \PYG{o}{\PYGZlt{}} \PYG{n}{b}
\PYG{n+nb+bp}{True}
\end{sphinxVerbatim}

\begin{sphinxVerbatim}[commandchars=\\\{\}]
\PYG{n}{a} \PYG{o}{=} \PYG{l+m+mi}{9}
\PYG{n}{b} \PYG{o}{=} \PYG{l+m+mi}{7}
\PYG{n}{a} \PYG{o}{\PYGZlt{}} \PYG{n}{b}
\PYG{n+nb+bp}{False}
\end{sphinxVerbatim}
\begin{itemize}
\item {} 
\textless{}= Küçük eşittir operatörü python örnek :

\end{itemize}

\begin{sphinxVerbatim}[commandchars=\\\{\}]
\PYG{n}{a} \PYG{o}{=} \PYG{l+m+mi}{5}
\PYG{n}{b} \PYG{o}{=} \PYG{l+m+mi}{7}
\PYG{n}{a} \PYG{o}{\PYGZlt{}}\PYG{o}{=} \PYG{n}{b}
\PYG{n+nb+bp}{True}
\end{sphinxVerbatim}

\begin{sphinxVerbatim}[commandchars=\\\{\}]
\PYG{n}{a} \PYG{o}{=} \PYG{l+m+mi}{7}
\PYG{n}{b} \PYG{o}{=} \PYG{l+m+mi}{7}
\PYG{n}{a} \PYG{o}{\PYGZlt{}}\PYG{o}{=} \PYG{n}{b}
\PYG{n+nb+bp}{True}
\end{sphinxVerbatim}
\begin{itemize}
\item {} 
== Eşittir operatörü python örnek :

\end{itemize}

\begin{sphinxVerbatim}[commandchars=\\\{\}]
\PYG{n}{a} \PYG{o}{=} \PYG{l+m+mi}{5}
\PYG{n}{b} \PYG{o}{=} \PYG{l+m+mi}{7}
\PYG{n}{a} \PYG{o}{==} \PYG{n}{b}
\PYG{n+nb+bp}{False}
\end{sphinxVerbatim}

\begin{sphinxVerbatim}[commandchars=\\\{\}]
\PYG{n}{a} \PYG{o}{=} \PYG{l+m+mi}{7}
\PYG{n}{b} \PYG{o}{=} \PYG{l+m+mi}{7}
\PYG{n}{a} \PYG{o}{==} \PYG{n}{b}
\PYG{n+nb+bp}{True}
\end{sphinxVerbatim}
\begin{itemize}
\item {} 
!= Eşit değil operatörü python örnek :

\end{itemize}

\begin{sphinxVerbatim}[commandchars=\\\{\}]
\PYG{n}{a} \PYG{o}{=} \PYG{l+m+mi}{5}
\PYG{n}{b} \PYG{o}{=} \PYG{l+m+mi}{7}
\PYG{n}{a} \PYG{o}{!=} \PYG{n}{b}
\PYG{n+nb+bp}{True}
\end{sphinxVerbatim}

\begin{sphinxVerbatim}[commandchars=\\\{\}]
\PYG{n}{a} \PYG{o}{=} \PYG{l+m+mi}{7}
\PYG{n}{b} \PYG{o}{=} \PYG{l+m+mi}{7}
\PYG{n}{a} \PYG{o}{!=} \PYG{n}{b}
\PYG{n+nb+bp}{False}
\end{sphinxVerbatim}
\begin{itemize}
\item {} 
\textgreater{}= Büyük eşit operatörü python örnek :

\end{itemize}

\begin{sphinxVerbatim}[commandchars=\\\{\}]
\PYG{n}{a} \PYG{o}{=} \PYG{l+m+mi}{5}
\PYG{n}{b} \PYG{o}{=} \PYG{l+m+mi}{7}
\PYG{n}{a} \PYG{o}{\PYGZgt{}}\PYG{o}{=} \PYG{n}{b}
\PYG{n+nb+bp}{False}
\end{sphinxVerbatim}

\begin{sphinxVerbatim}[commandchars=\\\{\}]
\PYG{n}{a} \PYG{o}{=} \PYG{l+m+mi}{7}
\PYG{n}{b} \PYG{o}{=} \PYG{l+m+mi}{7}
\PYG{n}{a} \PYG{o}{\PYGZgt{}}\PYG{o}{=} \PYG{n}{b}
\PYG{n+nb+bp}{True}
\end{sphinxVerbatim}
\begin{itemize}
\item {} 
\textgreater{} Büyüktür operatörü python örnek :

\end{itemize}

\begin{sphinxVerbatim}[commandchars=\\\{\}]
\PYG{n}{a} \PYG{o}{=} \PYG{l+m+mi}{5}
\PYG{n}{b} \PYG{o}{=} \PYG{l+m+mi}{7}
\PYG{n}{a} \PYG{o}{\PYGZgt{}} \PYG{n}{b}
\PYG{n+nb+bp}{False}
\end{sphinxVerbatim}

\begin{sphinxVerbatim}[commandchars=\\\{\}]
\PYG{n}{a} \PYG{o}{=} \PYG{l+m+mi}{9}
\PYG{n}{b} \PYG{o}{=} \PYG{l+m+mi}{7}
\PYG{n}{a} \PYG{o}{\PYGZgt{}} \PYG{n}{b}
\PYG{n+nb+bp}{True}
\end{sphinxVerbatim}
\end{quote}


\subsection{Email Alarm nedir?}
\label{\detokenize{email-sms-alert:email-alarm-nedir}}
Iot Kanal/Element alarm işlemi uygulandığında gelen değer ile alarm değeri mantıksal operatör işlem sonucuna göre kanal yöneticisine
email gönderilir. Kayıt olur iken kullanılan email adresi geçerli email adresidir. Günlük (24 saat) email gönderilme sayısı
üyelik planına göre hesaplanır.


\subsection{SMS Alarm nedir?}
\label{\detokenize{email-sms-alert:sms-alarm-nedir}}
Iot Kanal/Element alarm işlemi uygulandığında gelen değer ile alarm değeri mantıksal operatör işlem sonucuna göre kanal yöneticisine
sms gönderilir. Sms mesaj gönderilebilmesi için kanal yöneticisinin cep telefonunun onaylı olması gerekir. Günlük (24 saat) sms gönderilme sayısı
üyelik planına göre hesaplanır.


\section{Iot Mqtt Nedir?}
\label{\detokenize{what-is-iot-mqtt:what-is-iot-mqtt}}\label{\detokenize{what-is-iot-mqtt:iot-mqtt-nedir}}\label{\detokenize{what-is-iot-mqtt::doc}}
\sphinxstylestrong{MQTT} Message Queuing Telemetry Transport kelimelerinin baş harfleri ile tanıdığımız bu teknoloji
mesajın karşı tarafa ulaştırılması için kullanılan haberleşme protokolüdür.  Haberleşme için mesaj yayınlayan,
mesaja abone olan ve mesaj trafiğini kontrol eden yöneticiden oluşmaktadır.

Mesaj trafiğini kontrol eden yöneticiye BROKER, mesaj yayınına PUBLISH ve aboneye SUBSCRIBE denir. Mesaj alışverişi
publisher dan subscriber lara doğru yani yayıncılardan abonelere doğru olmaktadır.
\begin{figure}[htbp]
\centering
\capstart

\noindent\sphinxincludegraphics[scale=0.65]{{mqtt}.png}
\caption{\sphinxstyleemphasis{MQTT}}\label{\detokenize{what-is-iot-mqtt:id1}}\end{figure}


\section{Mqtt Protokolü Nasıldır?}
\label{\detokenize{what-is-iot-mqtt:mqtt-protokolu-nasildir}}
MQTT de asenkron haberleşme protokolü kullanılmaktadır. Mesaj yayıncıları ve mesaj alıcılar arasında
eşzamansız olarak veri taşınmaktadır. Diğer haberleşme yapılarına göre basit oluşu ve
minimum kaynak tüketmesi sebebiyle “machine-to-machine” (M2M)  makineden makineye veri
iletiminde ve (IOT) “Internet of Things” İnternete bağlı nesnelerin mesajlaşmasında tercih edilmektedir.


\section{MQTT Temp Test Client}
\label{\detokenize{mqtt-temp-test-client::doc}}\label{\detokenize{mqtt-temp-test-client:mqtt-temp-test-client}}\label{\detokenize{mqtt-temp-test-client:id1}}
Iot MQTT Temp Test Client
Mosquitto Brokera websocket ile gelen temp/random başlığını dinler.

Bu örnek `test.mosquitto.org' sitesinden alınmıştır. MQTT Temp örneğinin orjinal kaynağına `desert-home.com' adresinden ve Github üzerinden ulaşabilirsiniz.
MQTT Brokera nasıl mesaj gönderebilirim?

Iothook MQTT brokerına veri göndermek için \sphinxquotedblleft{}temp/random\sphinxquotedblright{} başlığı gönderilmelidir. Gönderilen değer -20 ile +50 aralığında kayar noktalı (float) veya tamsayı (int) formatında olmalıdır.

MQTT Broker kimlik doğrulama ile çalışır.

Örnek -\textgreater{} Mesaj yayınlama: mosquitto\_pub -h iothook.com -p 1883 -t \sphinxquotedblleft{}temp/random\sphinxquotedblright{} -m \sphinxquotedblleft{}6\sphinxquotedblright{} -u pub\_user -P iothook\_pub\_user

Örnek -\textgreater{} Mesaja abone olma: mosquitto\_sub -h iothook.com -p 1883 -t \sphinxquotedblleft{}temp/random\sphinxquotedblright{} -u pub\_user -P iothook\_pub\_user

MQTT Brokera için Test Kullanıcıları:

Kullanıcı Adı : pub\_user Şifre : iothook\_pub\_user

Kullanıcı Adı : sub\_user Şifre : iothook\_sub\_user

Kullanıcı Adı : pub\_client Şifre : iothook\_pub\_client

Kullanıcı Adı : sub\_client Şifre : iothook\_sub\_client
\begin{figure}[htbp]
\centering
\capstart

\noindent\sphinxincludegraphics[scale=0.65]{{mqtt-temp-test}.png}
\caption{\sphinxstyleemphasis{MQTT Temp Test}}\label{\detokenize{mqtt-temp-test-client:id2}}\end{figure}

Temp client sayfasına
\sphinxurl{https://iothook.com/mqtt/mqtt-temp-test/}
adresinden ulaşabilirsiniz.


\section{MQTT Test Client Publisher Subscriber}
\label{\detokenize{mqtt-temp-test-client-publisher-subscriber:mqtt-temp-test-client-publisher-subscriber}}\label{\detokenize{mqtt-temp-test-client-publisher-subscriber::doc}}\label{\detokenize{mqtt-temp-test-client-publisher-subscriber:mqtt-test-client-publisher-subscriber}}
MQTT Brokera Mesaj Gönderme ve Abone Olma
Mosquitto Brokera websocket ile gelen temp/random başlığını dinler.

Iothook MQTT brokerına veri göndermek için \sphinxquotedblleft{}temp/random\sphinxquotedblright{} başlığı gönderilmelidir. Gönderilen veri kayar
noktalı (float), tamsayı (int) veya string (text) formatında olabilir.

MQTT Broker kimlik doğrulama ile çalışır.

Örnek -\textgreater{} Mesaj yayınlama: mosquitto\_pub -h iothook.com -p 1883 -t \sphinxquotedblleft{}temp/random\sphinxquotedblright{} -m \sphinxquotedblleft{}6\sphinxquotedblright{} -u pub\_user -P iothook\_pub\_user

Örnek -\textgreater{} Mesaja abone olma: mosquitto\_sub -h iothook.com -p 1883 -t \sphinxquotedblleft{}temp/random\sphinxquotedblright{} -u pub\_user -P iothook\_pub\_user

MQTT Brokera için Test Kullanıcıları:

Kullanıcı Adı : pub\_user Şifre : iothook\_pub\_user

Kullanıcı Adı : sub\_user Şifre : iothook\_sub\_user

Kullanıcı Adı : pub\_client Şifre : iothook\_pub\_client

Kullanıcı Adı : sub\_client Şifre : iothook\_sub\_client
\begin{figure}[htbp]
\centering
\capstart

\noindent\sphinxincludegraphics[scale=0.65]{{mqtt-temp-test-client-publisher-subscriber}.png}
\caption{\sphinxstyleemphasis{MQTT Test Publisher Subscriber}}\label{\detokenize{mqtt-temp-test-client-publisher-subscriber:id1}}\end{figure}

Temp client sayfasına
\sphinxurl{https://iothook.com/mqtt/mqtt-temp-test-pub-sub/}
adresinden ulaşabilirsiniz.


\section{Full Featured MQTT Client}
\label{\detokenize{full-featured-mqtt-client::doc}}\label{\detokenize{full-featured-mqtt-client:full-featured-mqtt-client}}\label{\detokenize{full-featured-mqtt-client:id1}}
MQTT Brokera Mesaj Gönderme ve Alma

MQTT Broker kimlik doğrulama ile çalışır.

Örnek -\textgreater{} Mesaj yayınlama: mosquitto\_pub -h iothook.com -p 1883 -t “temp/random” -m “6” -u pub\_user -P iothook\_pub\_user

Örnek -\textgreater{} Mesaja abone olma: mosquitto\_sub -h iothook.com -p 1883 -t “temp/random” -u pub\_user -P iothook\_pub\_user

MQTT Brokera için Test Kullanıcıları:

Kullanıcı Adı : pub\_user Şifre : iothook\_pub\_user

Kullanıcı Adı : sub\_user Şifre : iothook\_sub\_user

Kullanıcı Adı : pub\_client Şifre : iothook\_pub\_client

Kullanıcı Adı : sub\_client Şifre : iothook\_sub\_client

Websockets Client Uygulaması Apache License Version 2.0 ile HiveMQ \sphinxurl{http://www.hivemq.com/} tarafından dağıtılmaktadır. Lisans hakkında ayrıca bilgi alınız.

Temp client sayfasına
\sphinxurl{https://iothook.com/mqtt/full-featured-mqtt-client/}
adresinden ulaşabilirsiniz.


\section{IHook Nedir?}
\label{\detokenize{what-is-ihook::doc}}\label{\detokenize{what-is-ihook:ihook-nedir}}\label{\detokenize{what-is-ihook:what-is-ihook}}

\subsection{Iot Dashboard Nedir?}
\label{\detokenize{what-is-ihook:iot-dashboard-nedir}}
Banana Pi, NanoPC, Intel Edison, Parallella, Raspberry Pi gibi tek kart bilgisayarlarda çalışan Python/Django REST framework ile geliştirilmiş Web Api servisidir. IOT cihazlar ile iletişime geçerek Web Api sayesinde GET, POST, PUT ve DELETE işlemlerini kolayca yapabilmek için tasarlanmıştır.
Iotdashboard tüm cihazlarınız arasında kesintisiz veri aktarımı yapan, internete bağlı nesnelerin kolayca ulaşabileceği iletişim protokollerini destekler. Google developer chart apileri ile entegre olarak verileri gerçek zamanlı izleme olanağı sağlar.
Proje iOTHOOK tarafından açık kaynak olarak geliştirilmiş ve MIT lisansı ile dağıtılmaktadır. Kaynak kodlara \sphinxurl{http://electrocoder.github.io/iotdashboard/} buradan  ulaşabilirsiniz.


\subsection{IHook GITHUB}
\label{\detokenize{what-is-ihook:ihook-github}}
Iot Dashboard GITHUB
Iot dashboard projesi Raspberry Pi türevi tek kart bilgisayarlar için geliştirilmiş Django Rest framewok server projesidir.
Proje iOTHOOK tarafından açık kaynak olarak geliştirilmiş ve MIT lisansı ile dağıtılmaktadır. Kaynak kodlara \sphinxurl{http://electrocoder.github.io/iotdashboard/} buradan ulaşabilirsiniz.



\renewcommand{\indexname}{Dizin}
\printindex
\end{document}